\documentclass[10pt]{article}

%------------------------------------------------------
%   PACKAGES
%------------------------------------------------------

% Default 
\usepackage{graphicx}
\usepackage[backend=biber,
  style=numeric, 
  sorting=none]{biblatex}

% Additional
\usepackage{amsmath}
\usepackage{textcomp, gensymb}
\usepackage{placeins}
\usepackage{tabularray} 
\usepackage{xcolor}
\usepackage{placeins}
\usepackage{todonotes}

\newcommand{\td}[1]{\todo[linecolor=blue, backgroundcolor=blue!25,bordercolor=blue, size=\small]{#1}}

\addbibresource{references.bib}

\title{Optical Activity} 
\author{Rahmanyaz Annyyev, Hikmat Gulaliyev}
\date{28 March 2024} 

\begin{document}

\maketitle

\begin{abstract}

This experiment investigates the phenomenon of optical activity, where certain molecules rotate the plane of polarization of light. The focus is on sugar solutions, which are optically active due to their chiral structure. The experiment utilizes a laser, polarizer, analyzer, and sugar solutions of varying concentrations. By measuring the rotation angle of the light after passing through the solution, the relationship between optical activity and concentration is explored. The specific rotation, a characteristic property of the sugar molecule, is not directly measured but can be calculated using the observed rotation angle, path length, and concentration. The experiment allows for the classification of the sugar solution as dextrorotatory (rotating the plane to the right) or levorotatory (rotating the plane to the left).

\end{abstract}

\section{Introduction}

\textit{Optical activity} is the ability of substances to rotate the plane of polarization of a linearly polarized beam of light about an optical axis as it passes through them. This phenomenon is due to the chiral (asymmetric) structure of certain molecules. Chirality is a property of molecules that are not superimposable on their mirror images, much like a left and right hand. The most common optically active substances are chiral organic compounds, such as sugars, amino acids, and certain drugs. Optical activity was first observed by French physicist Jean-Baptiste Biot in 1815 when he discovered that certain liquids rotated the plane of polarized light. His conclusions were experimentally supported by Louis Pasteur in the mid-19th century \cite{Petrucci_2017}.

The rotation of the plane polarization may be either to the right (clockwise) or to the left (counterclockwise). The former is called \textit{dextrorotation}, and the latter is called \textit{levorotation}. The angle of rotation is determined by the specific rotation of the substance, the concentration of the solution, and the path length of the light through the solution.

A linearly polarized light can be described as a superposition of two circularly polarized waves, one rotating clockwise and the other counterclockwise. When this light passes through an optically active substance, the two circularly polarized components travel at different speeds, leading to a phase difference between them. This phase difference results in a rotation of the plane of polarization of the light. Mathematically, this phenomenon can be described as follows:
\begin{equation}
    E_{\theta} = E_{\text{RHC}} + e^{2i\theta}E_{\text{LHC}},
\end{equation}
where $E_{\theta}$ is the electric field of the light after passing through the optically active substance, $E_{\text{RHC}}$ and $E_{\text{LHC}}$ are the electric fields of the right- and left-handed circularly polarized components, and $\theta$ is the rotation angle.

The specific rotation of a substance is a characteristic property that describes the angle of rotation per unit concentration and path length. It is denoted by the symbol $\alpha$ and is defined as:
\begin{equation}
    \alpha = \frac{\theta}{c \cdot l},
\end{equation}
where $\theta$ is the observed rotation angle, $c$ is the concentration of the solution, and $l$ is the path length of the light through the solution. The unit of specific rotation is degrees per decimeter per gram per milliliter ($\degree \text{dm}^{-1} \text{g}^{-1} \text{mL}$). The unit of degrees will be used in this experiment.


In this experiment, we will send a plane-polarized laser beam through a sugar solution and measure the rotation angle of the light after it passes through the solution. By varying the concentration of the sugar solution, we will explore the relationship between optical activity and concentration.

\section{Data \& Results}

\section{Discussion \& Conclusion}

\printbibliography

\end{document}