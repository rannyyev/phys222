\documentclass[10pt]{article}

% Default 
\usepackage{graphicx}
\usepackage[backend=biber,
  style=numeric, 
  sorting=none]{biblatex}

% Additional
\usepackage{amsmath}
\usepackage{textcomp, gensymb}
\usepackage{placeins}
\usepackage{tabularray} 
\usepackage{xcolor}
\usepackage{placeins}
\usepackage{csquotes} 
\usepackage{todonotes} 

\newcommand{\td}[1]{\todo[linecolor=blue, backgroundcolor=blue!25,bordercolor=blue, size=\small, inline]{#1}}

\addbibresource{references.bib}

\title{Thin Lens} 
\author{Rahmanyaz Annyyev, Hikmat Gulaliyev}
\date{29 March 2024} 

\begin{document}

\maketitle

\begin{abstract}
  Lenses are integral parts of most optical systems as they refract light rays to form images. In this experiment, the characteristics of thin lenses are explored through a series of experiments. The experiment consists of three parts, where in the first part thin lens equation is used to determine the focal length of a converging lens.
  The second part employs the Bessel method to verify the answer found in the previous part. Part C uses the virtual image method to determine the focal length of a diverging lens since it is not possible using previous methods. In general experiment was successful and the results are in line with manufacturer specifications. 
\end{abstract}

\section{Introduction}

In the most general sense, a \textbf{lens} is a refracting body that redirects the light rays passing through it. It either focuses or diverges the light rays, depending on the shape of the lens. A simple lens consists of a single piece of transparent material, whereas a compound lens consists of several simple lenses, called \textit{elements}, arranged along a common optical axis. Lenses are usually made of glass or plastic and are comprised of two surfaces, one or both of which are curved. The main purpose of a lens is to form an image of an object. The image can be real or virtual, depending on the position of the object relative to the lens. A real image is formed when the light rays converge at a point, whereas a virtual image is formed when the light rays diverge from a point \cite{Giancoli_2014}. 

\begin{figure}[hbt!]
  \centering
  \includegraphics[scale=0.5]{figures/f1.pdf}
  \caption{The converging lens.}
  \label{fig:1}
\end{figure}

The most common types of lenses are \textbf{converging} (convex) and \textbf{diverging} (concave) lenses. A converging lens is thicker in the middle than at the edges and focuses light rays to a point called the focal point. A diverging lens is thinner in the middle than at the edges and spreads light rays apart. The focal length of a lens is the distance between the lens and the focal point. The focal length of a converging lens is positive, whereas the focal length of a diverging lens is negative. The \textbf{thin lens equation} relates the object distance, the image distance, and the focal length of a lens. The equation is given by
\begin{equation}
  \label{eq:1}
  \frac{1}{f} = \frac{1}{s_{\text{o}}} + \frac{1}{s_{\text{i}}},
\end{equation}
where $f$ is the focal length of the lens, $s_{\text{o}}$ is the object distance, and $s_{\text{i}}$ is the image distance \cite{Pedrotti_2006}.  

\begin{figure}[hbt!]
  \centering
  \includegraphics[scale=0.5]{figures/f2.pdf}
  \caption{The diverging lens.}
  \label{fig:2}
\end{figure}

Lenses are categorized into two distinct types: thin lenses and thick lenses. A thin lens is defined by its negligible thickness in comparison to the radii of curvature of the lens surfaces. Conversely, a thick lens is characterized by a thickness that is not negligible in relation to the radii of curvature of the lens surfaces. This classification is dependent on the specific problem being addressed and the desired level of accuracy \cite{Hecht_2017}. 

The experiment consists of three parts: A, B, and C. The experimental setup is comprised of a $15$ cm converging lens, a $7.5$ cm diverging lens, a light source, a crossed arrow target that represents an object, and a screen on which the image is formed. The lenses will be assumed to be thin lenses. In each part of the experiment, we calculate the average focal length, $\bar{f}$, of the lens and the error, $\Delta \bar{f}$, in the measurement. The former is calculated using the formula
\begin{equation}
  \bar{f} = \dfrac{1}{n} \sum_{m=1}^{n} f_{m},
\end{equation}
where $n$ is the number of measurements and $f_{m}$ is the focal length of the lens in the $m$th measurement. The error is calculated using the formula
\begin{equation}
  \Delta \bar{f} = \sqrt{\dfrac{\sum_{m=1}^{n} (f_m - \bar{f})^2}{n-1}},
\end{equation}
where $f_m$ is the focal length of the lens in the $m$th measurement. The error is a measure of the uncertainty in the measurement of the focal length of the lens.

\subsection*{Part A}

\begin{figure}[hbt!]
  \centering
  \includegraphics[scale=0.5]{figures/f3.pdf}
  \caption{The experimental setup of part A.}
  \label{fig:3}
\end{figure}

In part A, we will employ the thin lens equation to determine the focal length of the converging lens. The setup is shown in Figure~\ref{fig:3}. The thin lens equation is given in Equation~\ref{eq:1}. First, we place the converging lens on the optical bench and adjust its position to obtain a sharp image of the object on the screen. We measure the object distance, $s_{\text{o}}$, from the lens to the object's position and the image distance, $s_{\text{i}}$, from the lens to the screen's position. We repeat this process two more times with different object distances.

\subsection*{Part B}

\begin{figure}[hbt!]
  \centering
  \includegraphics[scale=0.5]{figures/f4.pdf}
  \caption{The experimental setup of part B.}
  \label{fig:4}
\end{figure}

In part B, we will use the Bessel method to determine the focal length of the converging lens. The setup is shown in Figure~\ref{fig:4}. The formula used in the Bessel method is given by
\begin{equation}
  f = \dfrac{D^2 - d^2}{4D},
\end{equation}
where $f$ is the focal length of the lens, $D$ is the distance between the two positions of the screen, and $d$ is the distance between the two positions of the lens. Let's derive this formula before proceeding with the methodological explanation. Consider Figure~\ref{fig:4}. The object is placed at a distance $s_{\text{o}}$ from the lens. The image is formed at a distance $s_{\text{i}}$ from the lens. 

\begin{figure}[hbt!]
  \centering
  \includegraphics[scale=0.5]{figures/f5.pdf}
  \caption{The derivation of the Bessel method formula.}
  \label{fig:5}
\end{figure}

Let's denote the seperation between object and its image as $D = s_{\text{i}} + s_{\text{o}}$. The distance between the two positions of the lens is $d$. Then, $s_{\text{o}} = D - s_{\text{i}}$. If we substitute this into the thin lens equation, we get
\begin{equation}
  \dfrac{1}{f} = \dfrac{1}{D - s_{\text{i}}} + \dfrac{1}{s_{\text{i}}}.
\end{equation}
If we simplify this equation, we get
\begin{equation}
  s_{\text{i}}^2 - Ds_{\text{i}} + fD = 0.
\end{equation}
The discriminant of this equation is given by
\begin{equation}
  \Delta = D^2 - 4fD,
\end{equation}
and the focal length of the lens is given by
\begin{equation}
  f = \dfrac{D^2 - d^2}{4D}.
\end{equation}
Let's return to the experiment. First, we place the converging lens on the optical bench such that the object distance, $s_{\text{o}}$, is at least four times the estimated focal length of the lens. Next, we determine two positions of the lens that produce a sharp image of the object on the screen. We measure the distance between the two positions of the lens, $d$, and the distance between the two positions of the screen, $D$. We repeat this process two more times with different object distances.

\subsection*{Part C}

\begin{figure}[hbt!]
  \centering
  \includegraphics[scale=0.5]{figures/f6.pdf}
  \caption{The experimental setup of part C.}
  \label{fig:6}
\end{figure}

In part C, we will use the virtual object method to determine the focal length of the diverging lens. The setup is shown in Figure~\ref{fig:5}. It is not possible to directly measure the focal length of a diverging lens using the previous methods because the image formed by a diverging lens is always virtual. In the virtual object method, we use a converging lens to create a real image of the object, which then serves as a virtual object for the diverging lens. The converging lens should be sufficiently \enquote{strong} to satisfy the condition 
\begin{equation}
  \dfrac{1}{f_{\text{c}}} > \left| \dfrac{1}{f_{\text{d}}} \right|,
\end{equation}
where $f_{\text{c}}$ is the focal length of the converging lens and $f_{\text{d}}$ is the focal length of the diverging lens. If this condition is not satisfied, the image formed by the converging lens will be virtual, and the virtual object method will not work. The formula used in the virtual object method is given by
\begin{equation}
  f = \dfrac{s^{'}_{\text{i}} \left( s_{\text{i}} - d \right)}{s_{\text{i}} - \left( s^{'}_{\text{i}} + d \right)},
\end{equation}
where $f$ is the focal length of the diverging lens, $s_{\text{i}}$ is the distance between the first position of the screen and the converging lens, $s^{'}_{\text{i}}$ is the distance between the second position of the screen and the diverging lens, and $d$ is the distance between the lenses. First, we mount the converging lens on the optical bench such that the object distance, $s_{\text{o}}$, is in the range of $10$ to $12$ cm. Then, we adjust the position of the screen to obtain a sharp image of the object. We measure the image distance, $s_{\text{i}}$, from the lens to the screen's first position. Next, we place the diverging lens on the optical bench and adjust the screen to obtain a sharp image of the object. We measure the image distance, $s'_{\text{i}}$, from the lens to the screen's second position. We measure the distance between the two positions of the screen, $d$. We repeat this process two more times with different object distances.


\section{Data \& Results}

\subsection*{Part A}

The results of part A are shown in Table~\ref{tab:1}. The average focal length of the converging lens is $7.50$ cm with an error of $0.13$ cm. It is within the limits of error of the actual focal distance of the lens, which corresponds to $7.5$ cm.

\begin{table}[ht]
  \centering
  \vspace{4mm}
  \begin{tblr}{
    cells = {halign = c, valign = m},
    row{odd} = {bg = lightgray!5},
    row{1} = {bg = lightgray!20},
    hlines = {},
    vlines = {},
    cell{5}{2}={c=2}{c},
    cell{6}{2}={c=2}{c}
  }
    $s_{\text{o}}$ (cm) & $s_\text{{i}}$ (cm) & Focal length, $f_{\text{m}}$ (cm) \\
    \hline
    18 & 12.5 & 7.38 \\
    28 & 10.5 & 7.63 \\
    19.5 & 12.2 & 7.50 \\
    \hline
    Average focal length, $\bar{f}$ (cm) & Error, $\Delta \bar{f}$ (cm) & \\
    7.50 & 0.13 \\ 
  \end{tblr}
  \caption{Results of part A of the experiment.}
  \label{tab:1}
\end{table}

\subsection*{Part B}

The results of part B are shown in Table~\ref{tab:2}. The average focal length of the converging lens is $7.28$ cm with an error of $0.02$ cm. It is not within the limits of error of the actual focal length of the lens, which is $7.5$ cm. The average value of the focal length in part A and part do not agree with each other within the limits of error.

\begin{table}[ht]
  \centering
  \vspace{4mm}
  \begin{tblr}{
    cells = {halign = c, valign = m},
    row{odd} = {bg = lightgray!5},
    row{1} = {bg = lightgray!20},
    hlines = {},
    vlines = {},
    cell{5}{2}={c=2}{c},
    cell{6}{2}={c=2}{c}
  }
    $d$ (cm) & $D$ (cm) & Focal length, $f_{\text{m}}$ (cm) \\
    \hline
    20.5 & 39.7 & 7.28 \\
    24.5 & 43 & 7.26 \\
    11.2 & 33 & 7.30 \\
    \hline
    Average focal length, $\bar{f}$ (cm) & Error, $\Delta \bar{f}$ (cm) & \\
    7.28 & 0.02 \\ 
  \end{tblr}
  \caption{Results of part B of the experiment.}
  \label{tab:2}
\end{table}

\subsection*{Part C}

The results of part C are shown in Table~\ref{tab:3}. The average focal length of the diverging lens is $-15.1$ cm with an error of $0.95$ cm. It is within the limits of error of the actual focal length of the lens, which is $-15$ cm.

\begin{table}[ht]
  \centering
  \vspace{4mm}
  \begin{tblr}{
    cells = {halign = c, valign = m},
    row{odd} = {bg = lightgray!5},
    row{1} = {bg = lightgray!20},
    hlines = {},
    vlines = {},
    cell{5}{2}={c=3}{c},
    cell{6}{2}={c=3}{c}
  }
    $s_{\text{i}}$ (cm) & $s'_{\text{i}}$ (cm) & d (cm) & Focal length, $f_{\text{m}}$ (cm) \\
    \hline
    16.4 & 22.8 & 7 & -16.0 \\
    18 & 25 & 9 & -14.1 \\
    21 & 46 & 9.6 & -15.2 \\
    \hline
    Average focal length, $\bar{f}$ (cm) & Error, $\Delta \bar{f}$ (cm) & \\
    -15.1 & 0.95 \\ 
  \end{tblr}
  \caption{Results of part C of the experiment.}
  \label{tab:3}
\end{table}

\section{Discussion \& Conclusion}

There are a few sources of errors per the analog nature of this experiment, such as the parallax error, the subjective nature of the sharpness of the image, and the alignment of the lenses. The main approximations taken into account were uniformity and of the lens, the refractive index of the air being $n = 1$, and most importantly, the thin lens approximation.
The experiment was successful and the results are in line with the manufacturer's specifications in all parts but B where the observed discrepancy might be caused by limitations inherent to the experimental design, such as measurement accuracy and lens alignment. Since the sharpness of the image is subjective, different
people may have different opinions about what constitutes a sharp image. This
can lead to variations in the measurements of the object and image distances.

\section{Extra credit}


Lenses are a ubiquitous part of our daily lives. They are used in cameras, microscopes, telescopes, and many other devices. In this experiment, we have learned how to determine the focal length of a lens using the thin lens equation, the Bessel method, and the virtual object method. We have also learned how to calculate the magnification of a lens.
One of the most influential uses of lenses is prescription glasses and lenses. They affect more than 60\% of the world's population\cite{Parker}.

However, the usage of lenses is not limited to glasses. In the past few years, they have also become an integral part of military technology. Since the advancement of drones and laser-guided missile technologies, lenses have became more and more import in the military. 
\printbibliography

\end{document}