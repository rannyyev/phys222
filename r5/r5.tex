\documentclass[10pt]{article}

%------------------------------------------------------
%   PACKAGES
%------------------------------------------------------

% Default 
\usepackage{graphicx}
\usepackage[backend=biber,
  style=numeric, 
  sorting=none]{biblatex}

% Additional
\usepackage{amsmath}
\usepackage{textcomp, gensymb}
\usepackage{placeins}
\usepackage{tabularray} 
\usepackage{xcolor}
\usepackage{placeins}
\usepackage{todonotes}

\newcommand{\td}[1]{\todo[linecolor=blue, backgroundcolor=blue!25,bordercolor=blue, size=\small]{#1}}

\addbibresource{references.bib}

\title{Thin Lens} 
\author{Rahmanyaz Annyyev, Hikmat Gulaliyev}
\date{29 March 2024} 

\begin{document}

\maketitle

\begin{abstract}

\end{abstract}

\section{Introduction}



The experiment is comprised of three parts: A, B, and C. The experimental setup is comprised of a $150$ mm converging lens, a $75$ mm diverging lens, a light source, a crossed arrow target that represents an object, and a screen on which the image is formed.

In part A, we will employ the thin lens equation to determine the focal length of the converging lens. The setup is shown in Figure~. First...

In part B, we will use the Bessel method to determine the focal length of the converging lens. The setup is shown in Figure~. First...

In part C, we will use the virtual object method to determine the focal length of the diverging lens. The setup is shown in Figure~. First...

\section{Discussion \& Conclusion}

\section{Data \& Results}

The results of part A are shown in Table~\ref{tab:1}. 

\begin{table}[ht]
  \label{tab:1}
  \centering
  \vspace{4mm}
  \begin{tblr}{
    cells = {halign = c, valign = m},
    row{odd} = {bg = lightgray!5},
    row{1} = {bg = lightgray!20},
    hlines = {},
    vlines = {},
    cell{5}{2}={c=2}{c},
    cell{6}{2}={c=2}{c}
  }
    $s_{\text{o}}$ (cm) & $s_\text{{i}}$ (cm) & Focal length, $f_{\text{m}}$ (cm) \\
    \hline
    -- & -- & -- \\
    -- & -- & -- \\
    -- & -- & -- \\
    \hline
    Average focal length, $\bar{f}$ (cm) & Error, $\Delta \bar{f}$ (cm) & \\
    -- & -- \\ 
  \end{tblr}
  \caption{Results of part A of the experiment.}
\end{table}

The results of part B are shown in Table~\ref{tab:2}.

\begin{table}[ht]
  \label{tab:2}
  \centering
  \vspace{4mm}
  \begin{tblr}{
    cells = {halign = c, valign = m},
    row{odd} = {bg = lightgray!5},
    row{1} = {bg = lightgray!20},
    hlines = {},
    vlines = {},
    cell{5}{2}={c=2}{c},
    cell{6}{2}={c=2}{c}
  }
    $d$ (cm) & $D$ (cm) & Focal length, $f_{\text{m}}$ (cm) \\
    \hline
    -- & -- & -- \\
    -- & -- & -- \\
    -- & -- & -- \\
    \hline
    Average focal length, $\bar{f}$ (cm) & Error, $\Delta \bar{f}$ (cm) & \\
    -- & -- \\ 
  \end{tblr}
  \caption{Results of part B of the experiment.}
\end{table}

The results of part C are shown in Table~\ref{tab:3}.

\begin{table}[ht]
  \label{tab:3}
  \centering
  \vspace{4mm}
  \begin{tblr}{
    cells = {halign = c, valign = m},
    row{odd} = {bg = lightgray!5},
    row{1} = {bg = lightgray!20},
    hlines = {},
    vlines = {},
    cell{5}{2}={c=3}{c},
    cell{6}{2}={c=3}{c}
  }
    $s_{\text{i}}$ & $s'_{\text{i}}$ (cm) & d (cm) & Focal length, $f_{\text{m}}$ (cm) \\
    \hline
    -- & -- & -- \\
    -- & -- & -- \\
    -- & -- & -- \\
    \hline
    Average focal length, $\bar{f}$ (cm) & Error, $\Delta \bar{f}$ (cm) & \\
    -- & -- \\ 
  \end{tblr}
  \caption{Results of part C of the experiment.}
\end{table}

\section{Extra credit}

Lenses are a ubiquitous part of our daily lives. They are used in cameras, microscopes, telescopes, and many other devices. In this experiment, we have learned how to determine the focal length of a lens using the thin lens equation, the Bessel method, and the virtual object method. We have also learned how to calculate the magnification of a lens.

\printbibliography

\end{document}