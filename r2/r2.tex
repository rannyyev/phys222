\documentclass[10pt]{article}

%------------------------------------------------------
%   PACKAGES
%------------------------------------------------------

% Default 
\usepackage{graphicx}
\usepackage[backend=biber,style=numeric,sorting=ynt]{biblatex}

% Additional
\usepackage{amsmath}
\usepackage{textcomp, gensymb}
\usepackage{placeins}
\usepackage{tabularray} 
\usepackage{xcolor}
\usepackage{placeins}
\usepackage{todonotes}
\newcommand{\td}[1]{\todo[linecolor=blue, backgroundcolor=blue!25,bordercolor=blue, size=\small]{#1}}

\addbibresource{references.bib}

\title{Polarization} 
\author{Rahmanyaz Annyyev, Hikmat Gulaliyev} 
\date{14 March 2024} 

\begin{document}

\maketitle

\begin{abstract}

An electromagnetic (EM) wave is a disturbance that propagates in the electromagnetic field. It consists of two components, electric and magnetic fields, oscillating perpendicular to each other and the direction of the propagation. The polarization of an EM wave is the orientation of the electric field vector as the wave progresses in space. A polarized wave is one in which the electric field vector traces a specific pattern along the plane perpendicular to the direction of the propagation. Natural sources of electromagnetic radiation emit unpolarized light, that is, radiation in which the electric field oscillates in all possible directions. A polarizer is an optical device capable of polarizing the radiation. There are three types of polarization: linear, circular, and elliptical. Light is also an EM wave; hence, it is subject to polarization. In this experiment, we verify Malu's law by employing two polarizers, one of which serves as a polarizer, and the other as an analyzer. The intensity of the light after passing the analyzer is measured as a function of the angle between the axes of the polarizer and the analyzer. The results are consistent with Malu's law. The type of polarization is linear.

\end{abstract}

\section{Introduction}

A wave is a disturbance of a continuous medium that propagates with a fixed shape at constant velocity. If the medium is the \textit{electromagnetic field}, the wave is called an \textit{electromagnetic wave}. 

An electromagnetic wave is comprised of two synchronized oscillations of electric and magnetic fields. Both fields are perpendicular to each other and to the direction of the propagation. Hence, an electromagnetic wave is a \textit{transverse wave}.

An electromagnetic wave can be represented by two perpendicular components. If there is a well-defined relationship between the two components, the wave is said to be \textit{polarized}. An electromagnetic wave that is not polarized is called \textit{unpolarized wave}. The electric field of an unpolarized wave oscillates in all planes perpendicular to the direction of the propagation; it is a mixture of waves with different polarizations. Such a wave can be polarized by passing through a special filter called a \textit{polarizer}.

Essentially, polarization is the phenomenon of restricting, or fixing the oscillation of the electric field of light to a single plane. There are several types of polarization:
\begin{itemize}
  \item Linear polarization. The electric field vector traces a straight line.
  \item Circular polarization. The electric field vector traces a circle.
  \item Elliptical polarization. The electric field vector traces an ellipse.
\end{itemize}

Light is one of the types of electromagnetic waves, and in this experiment, we will study its linear polarization. The primary goal of the experiment is to verify the Malu's law. The equipment is comprised of two polarizers, a laser, a photodetector, an interface, and a rotary motion sensor. The first polarizer will produce linearly polarized light by allowing only the light oscillating in the vertical plane to pass through. The second polarizer will be used to measure the intensity of the light after passing through the first polarizer; it serves as an analyzer.

\begin{figure}[ht]
  \centering
  \includegraphics[scale=0.6]{figures/f1.pdf}
  \caption{Experimental setup.}
  \label{fig:1}
\end{figure}

Let's denote the angle between the axis of the first polarizer and the axis of the analyzer as $\theta$. If the intensity of the light before passing the analyzer is $I_0$, and the intensity of the light after passing the analyzer is $I$, then Malu's law states that the intensity of the light after passing the analyzer is given by the equation \cite{Hecht_2017}:

\begin{equation}
  I(\theta)_{out} = I_{in}\left[H_{90}+\left(H_0-H_{90}\right)\right]\cos^2\theta
  \label{eq:1}
\end{equation}

The procedure of the experiment is as follows. First, we turn on the laser and adjust the polarizers so that their axes are parallel. Next, we turn on the interface, namely Pasco\textsuperscript{\textregistered} Xplorer GLX, and start the data collection. We rotate the analyzer by 400\degree. Then we stop the data collection and conclude the experiment.

\section{Data \& Results}

After extracting and processing the data collected in the experiment, we obtained the plot intensity of the light after passing the analyzer as a function of the angle between the axes of the polarizer and the analyzer (see Figure~\ref{fig:2}).
The plot appears to be in the form of a sinusoidal function. Performing the best fit of the data (see Figure~\ref{fig:3}), a line of the form $I(\theta) = a + b\cos^2(\theta)$ was obtained:

\begin{equation}
  I(\theta) = 2.3271\cos^2(1.001\theta+1.42)
  \label{eq:2}
\end{equation}

\begin{figure}[h!]
  \centering
  \includegraphics[scale=0.6]{plots/plot1.png}
  \caption{Plot of the intensity of the polarized light versus the angle between the axes of the polarizer and the analyzer.}
  \label{fig:2}
\end{figure}

\begin{figure}[h!]
  \centering
  \includegraphics[scale=0.6]{plots/plot2.png}
  \caption{Best-fit of the data to Malu's law.}
  \label{fig:3}
\end{figure}

This suggests that Malu's law is consistent with the data, and the light is linearly polarized. 
 
\section{Discussion \& Conclusion}

The equipment used to collect the data was a digital interface that minimized the error in the measurements. However, there are still some sources of error. The phase shift in the best-fit line is due to the manual alignment of the polarizers at the start of the experiment. The fluctuations in the plot are due to the manual rotation of the analyzer, which was not uniform. The higher density of points in some regions than others is also due to the speed of rotation of the analyzer.

Another source of error is the best-fit plot. It is not the exact representation of the data. However, due to the large number of data points, we can assume that it is close to the real value.

In conclusion, the experiment was successful in verifying Malu's law, and also in determining that the incident light is linearly polarized.

\printbibliography

\end{document}