\documentclass[10pt]{article}

% Default 
\usepackage{graphicx}
\usepackage[
  backend=biber,
  style=numeric, 
  sorting=none
]{biblatex}

% Additional
\usepackage{amsmath}
\usepackage{textcomp, gensymb}
\usepackage{placeins}
\usepackage{tabularray} 
\usepackage{xcolor}
\usepackage{placeins}
\usepackage{csquotes} 
\usepackage{todonotes}
\usepackage{hyperref}
\usepackage{siunitx}

\newcommand{\td}[1]{\todo[linecolor=blue, backgroundcolor=blue!25,bordercolor=blue, size=\small, inline]{#1}}

\addbibresource{references.bib}

\title{Reflection} 
\author{Rahmanyaz Annyyev, Hikmat Gulaliyev}
\date{16 May, 2024} 

\begin{document}

\maketitle

\begin{abstract}

\end{abstract}

\section{Introduction}

\subsection*{General}

\subsection*{Procedure} 
The setup consists of an optical bench, two polarizers, an angular translator, a light-sensitive detector, a laser, a slit collimator, an interface, and a square analyzer.

The purpose of the experiment is to find refraction angles and intensities of parallel and perpendicular components of light at different incident angles.

Before starting the experiment setup needs to be calibrated. To do so we first need to mount the slit collimator away from the laser. After putting the slit collimator to configuration $\#5$, we set the angle of the angle translator to $0\degree$. Now, we can align the laser using its alignment screws to make sure the laser beam passes through the center of the detector aperture. 

After mounting polarizers on and setting the angle of polarizer $\#2$ to $45\degree$ to divide light into two linear electric field components that are perpendicular to each other, we can turn the interface on. Selecting the Light Sensor, and opening the METER icon, enables us to measure light intensity. However, the interface needs to be calibrated as well, by resetting the meter, and later adjusting the intensity of polarizer $\#1$ so that the maximum reading is not more than $100\%$

We can now put the cylindrical lens on the pivot plate and align it with the angular translator and pivot plane so that the flat surface faces the laser beam. Rotating the plates will now result in reflection and refraction at the dielectric interface that is formed between air and the flat surface.

Now we can set incident angle at different values by turning the angular translator, and find the refraction angle by again rotating the angular translator so that intensity is maximum. After noting the angle at Table 1. We can put in the square analyzer to also record horizontal and vertical components of light. Also by finding points where horizontal intensity is zero we can find Brewster's angle.

Now, we can 
\section{Data \& Results}

\subsection*{Part A}

\subsection*{Part B}

\section{Discussion \& Conclusion}

\subsection*{Errors}

\subsection*{Discrepancies}

\subsection*{Conclusion} 

\section{Extra Credit}

\printbibliography

\end{document}