\documentclass[10pt]{article}
\usepackage{graphicx}
\usepackage[backend=biber,style=numeric,sorting=ynt]{biblatex}
\usepackage{amsmath, gensymb}

\addbibresource{references.bib}

\title{Prism Spectrometer} 
\author{Rahmanyaz Annyyev, Hikmat Gulaliyev} 
\date{02 March 2024} 

\begin{document}

\maketitle

\begin{abstract}
In this experiment, we utilize a prism spectrometer---an optical device that separates light into its constituent frequencies---to measure the refractive index of a prism. A collimator with a slit is used to produce a parallel beam of light emitted by a mercury lamp, which is then passed through the prism. The light is then refracted and dispersed into its constituent frequencies. The angle of deviation of the light is measured and used to calculate the refractive index of the prism using a special relation. A graph of the index of refraction versus wavelength is also plotted. The results are compared to the theoretical values and the limitations of the experiment are discussed.
\end{abstract}

\section{Introduction}
The visible spectrum is the portion of electromagnetic radiation that is visible to the human eye. It is composed of light with wavelengths between 380 and 780 nm. Each wavelength, or frequency of light is perceived as a different color by the human eye\cite{Marcus_1998}. 



A prism spectrometer is an optical device used in spectrum analysis. It is composed of a collimator, a prism, and a telescope.

The collimator is a device that produces a parallel beam of light. It is composed of an adjustable slit on one end and a lens on the other. The slit is used to control the width of the beam of light, and the lens is used to focus the light into a parallel beam. 

The prism is used to refract and disperse the light into its constituent frequencies. The angle of deviation of the light is measured and used to calculate the refractive index of the prism using a special relation.

Snell's law is given by
\begin{equation}
    n_1 \sin(\theta_1) = n_2 \sin(\theta_2),
\end{equation}
where $n_1$ and $n_2$ are the refractive indices of the two media, and $\theta_1$ and $\theta_2$ are the angles of incidence and refraction, respectively

Calculations for table 1:


Calculations for table 2:
\begin{align}
    \delta_{min} &= 360.00\degree-351.80\degree+29.82\degree = 38.02\degree & \text{Violet} \\
    \delta_{min} &= 360.00\degree-351.80\degree+29.51\degree = 38.31\degree & \text{Blue} \\
    \delta_{min} &= 360.00\degree-351.80\degree+27.63\degree = 40.19\degree & \text{Green} \\
    \delta_{min} &= 360.00\degree-351.80\degree+28.64\degree = 39.18\degree & \text{Orange}
\end{align}

And the refractive index is given by
\begin{equation}
    n = \dfrac{\sin\left(\dfrac{\delta_{min}+\alpha}{2}\right)}{\sin\left(\dfrac{\alpha}{2}\right)},
\end{equation}


THe angle of prism is given by
\begin{equation}
    \alpha = \dfrac{\text{position 1}+\text{position 2}}{2}
\end{equation}

\begin{table}[ht]
    \centering
    \begin{tabular}{|c|c|c|c|c|}
        \hline 
        Position \#1 & Position \#2 & $\alpha_i$ \\
        \hline
        $52\degree 46'$ & $293\degree 01'$ & $59.86\degree$ \\
        \hline
        $52\degree 52'$ & $293\degree 05'$ & $59.88\degree$ \\
        \hline
        $52\degree 48'$ & $293\degree 03'$ & $59.86\degree$ \\
        \hline
        & & $\bar{\alpha} = 59.87\degree$ \\
        \hline
    \end{tabular}
    \caption{Data for the prism angle, $\alpha$.}
    \label{tab:1}
\end{table}



\begin{table}[ht]
    \centering
    \begin{tabular}{|c|c|c|c|c|}
        \hline
        $\lambda$ (nm) & Position \#1 & Position \#2 & $\delta_{min}$ & $n$ \\
        \hline
        $404.6$ Violet & $351\degree 48'$ & $29\degree 49'$ & $38.02$ & \\
        \hline
        $435.8$ Blue & $351\degree 48'$ & $29\degree 28'$ & $38.31$ & \\
        \hline
        $546.1$ Green & $351\degree 48'$ & $27\degree 40'$ & $40.19$ & \\
        \hline
        $579.1$ Orange & $351\degree 48'$ & $28\degree 41'$ & $39.18$ & \\
        \hline
    \end{tabular}
    \caption{Data for the refractive index versus wavelength.}
    \label{tab:2}
\end{table}



\section{Discussion \& Conclusion}

\printbibliography

\end{document}