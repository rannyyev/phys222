\documentclass[10pt]{article}

% Default 
\usepackage{graphicx}
\usepackage[
  backend=biber,
  style=numeric, 
  sorting=none
]{biblatex}

% Additional
\usepackage{amsmath}
\usepackage{textcomp}
\usepackage{gensymb}
\usepackage{placeins}
\usepackage{tabularray} 
\usepackage{xcolor}
\usepackage{placeins}
\usepackage{csquotes} 
\usepackage{todonotes}
\usepackage{hyperref}
\usepackage{siunitx}

\newcommand{\td}[1]{\todo[linecolor=blue, backgroundcolor=blue!25,bordercolor=blue, size=\small, inline]{#1}}

\addbibresource{references.bib}

\title{Diffraction and Interference} 
\author{Rahmanyaz Annyyev, Hikmat Gulaliyev}
\date{23 May, 2024} 

\begin{document}

\maketitle

\begin{abstract}

\td{Please expand more on the abstract to hit 200 words and include the information from the data calculation and discussion sections that you will write.}

The primary goal of this experiment is to examine the interference and diffraction phenomena when light passes through disparate apertures. The experiment is divided into two parts: part A and part B. In part A, the wavelength of the laser light is computed using a single-slit disk. In part B, the wavelength of the laser light is computed using a double-slit disk. The results of the experiment are presented in the tables below. The errors in the measurements are also calculated. The experiment makes several approximations, including the Fraunhofer diffraction pattern and coherent light. The discrepancies between the theoretical and experimental values are discussed. The experiment is concluded by summarizing the results and discussing the possible sources of error.

\end{abstract}

\section{Introduction}

\subsection*{General}

The history bears witness to the fact that the nature of light has been a subject of debate for centuries. The wave theory of light, which posits that light is a wave, was first proposed by Christiaan Huygens in the 17th century. This theory was later supported by Thomas Young's double-slit experiment in the early 19th century. The wave theory of light was further developed by James Clerk Maxwell in the 19th century, who showed that light is an electromagnetic wave. The wave theory of light has been confirmed by numerous experiments, including the diffraction and interference of light.

Interference is a phenomenon that occurs when two or more coherent waves overlap. Coherent waves are waves that have the same frequency, wavelength, and constant relative phase. When coherent waves overlap, they interfere constructively or destructively, creating a pattern of light and dark regions on a screen. This pattern is known as an interference pattern.

Diffraction is a phenomenon similar to interference that occurs when light passes through a small aperture with dimensions on the order of the wavelength of light. When light passes through a small aperture, it diffracts, that is, it bends and spreads out in all directions. This creates a pattern of light and dark regions on a screen. This pattern is known as a diffraction pattern.

Both interference and diffraction can be explained by the Huygens-Fresnel principle, which states that each point on a wavefront acts as a source of secondary spherical wavelets. The interference of these secondary wavelets produces the diffraction and interference patterns observed in experiments.

The relative distances between the light source, the aperture, and the screen determine whether the diffraction pattern is in the near field or the far field. In the near field, the diffraction pattern is complex and varies with distance, and it is called the Fresnel diffraction pattern. In the far field, the diffraction pattern is simpler and does not vary with distance, since the secondary wavelets are approximately plane waves, and it is called the Fraunhofer diffraction pattern.

\subsubsection*{Single-Slit Diffraction}

The intensity of the light in the Fraunhofer diffraction pattern produced by a single slit is given by the equation
\begin{equation}
  I(\theta) = I_0 \left( \frac{\sin(\alpha)}{\alpha} \right)^2
  \label{eq:single-slit-intensity}
\end{equation}
where
\begin{equation}
  \alpha = \frac{\pi a \sin(\theta)}{\lambda}
\end{equation}
and $I_0$ is the intensity of the light at the center of the pattern, $a$ is the width of the slit, $\theta$ is the observation angle, and $\lambda$ is the wavelength of the light. The zeros of the diffraction pattern occur at those angles where $\sin(\alpha) = 0$, that is, at
\begin{equation}
  \sin(\theta) = \frac{n \lambda}{a} \quad \text{for} \quad n = \pm 1, \pm 2, \pm 3, \ldots
\end{equation}
One of our assumptions is that the width of the slit is much smaller than the distance from the slit to the screen, that is, $a \ll L$. In this case, $\sin(\theta) \approx \tan(\theta) = y/L$, where $y$ is the distance from the center of the pattern to the $n$th minimum. Therefore, the wavelength of the light can be computed using the equation
\begin{equation}
  \lambda = \frac{a \bar{y}}{L}
\end{equation}
where $\bar{y}$ is the average distance between adjacent minima. 

\subsubsection*{Double-Slit Interference}

The intensity of the light in the Fraunhofer diffraction pattern produced by a double slit is when ignoring the effect of the slits is given by the equation
\begin{equation}
  I(\theta) = 4 I_0 \cos^2(\beta)
  \label{eq:double-slit-intensity-1}
\end{equation}
where
\begin{equation}
  \beta = \frac{\pi d \sin(\theta)}{\lambda}
\end{equation}
and $d$ is the distance between the slits, $\theta$ is the observation angle, and $\lambda$ is the wavelength of the light. It is assumed that the intensity produced by each slit is the same, that is, $I_1 = I_2 = I_0$. However, the intensity from a single-slit diffraction depends on the angle of observation. If we combine Equations \eqref{eq:single-slit-intensity} and \eqref{eq:double-slit-intensity-1}, we obtain the intensity of the light in the Fraunhofer diffraction pattern produced by a double slit as
\begin{equation}
  I(\theta) = I_0 \left( \frac{\sin(\alpha)}{\alpha} \right)^2 \cos^2(\beta)
  \label{eq:double-slit-intensity-2}
\end{equation}
The interference minimuma occur when
\begin{equation}
  \beta = (m + 1/2) \pi \quad \text{for} \quad m = 0, 1, 2, 3, \ldots
\end{equation}
and the maxima occur when
\begin{equation}
  \alpha = m \pi \quad \text{for} \quad m = 0, 1, 2, 3, \ldots
\end{equation}  
We will once again assume that the width of the slits is much smaller than the distance from the slits to the screen, that is, $d \ll L$. In this case, $\sin(\theta) \approx \tan(\theta) = y/L$, where $y$ is the distance from the center of the pattern to the $m$th minimum. Therefore, the wavelength of the light can be computed using the equation
\begin{equation}
  \lambda = \frac{d \bar{y}'}{L}
\end{equation}
where $\bar{y}'$ is the average distance between adjacent minima or maxima.

\subsection*{Procedure} 

The experiment is comprised of two parts: part A and part B. The setup consists of a diode laser, a light sensor, a linear translator, single and double-slit disks, a rotary motion sensor, an aperture bracket, and a PASCO{\textsuperscript\textregistered} data logger. As a preliminary step, we will spend some time to familiarize ourselves with various diffraction patterns caused by different slit widths and combinations of near and far field regions.

\subsubsection*{Part A}

The purpose of part A is to compute the wavelength of the laser light using a single-slit disk.

First, we set up the components for the Fraunhofer diffraction pattern. The light sensor is placed on the aperture bracket, which is then mounted on the linear translator. The rotary motion sensor is attached to the linear translator. The single-slit disk is placed in front of the laser; both are placed at a certain distance from the light sensor. The whole setup rests on an optical bench.

Next, the laser is turned on, and the disk is rotated until the slit pattern is in line with the beam of the laser. If necessary, the adjustment screws on the back of the laser may be used to align the beam with the slit. 

Then, the pulley on top of the rotary motion sensor is rotated to move the sensor along the linear translator until a diffraction pattern appears on the white screen attached to the aperture bracket. 

After the pattern is visible, we have to ensure that it is horizontal and centered on the screen. The screws on the slit accessory can be used to obtain the desired pattern, and the accessory may be rotated to adjust the pattern's orientation.

Next, the aperture disk attached in front of the bracket is rotated to block ambient light and to ensure that the sensor is between maxima of the diffraction pattern. For the best results, the narrowest slit should be used.

Then, the light sensor is moved along the linear translator to align the center of the diffraction pattern with the slit on the aperture disk. The height of the sensor may be adjusted using the rod clamp on the rotary motion sensor.

Thereafter, the PASCO{\textsuperscript\textregistered} data logger is used to record the intensity of the light at different positions along the linear translator. The vertical axis of the graph should be set to the intensity of the light (light sensor output), and the horizontal axis should be set to the position of the sensor (rotary motion sensor output).

The procedure is repeated for different slit widths, and the data is recorded for each slit width in the corresponding table.

\subsubsection*{Part B}

The purpose of part B is to compute the wavelength of the laser light using a double-slit disk. The procedure is the same as in part A.

\section{Data \& Results}

\subsection*{Part A}

\begin{table}[ht]
  \centering
  \begin{tblr}{
    cells = {halign = c, valign = m},
    column{1} = {halign = l},
    row{odd} = {bg = lightgray!5},
    hlines = {},
    vlines = {}
  }
    Pattern & A & B & C \\
    \hline 
    Width of the slit, $a$ (\si{\mm}) & 0.16 & 0.008 & 0.04 \\
    Distance from the slit to the screen, $L$ (\si{cm}) & 84.5 & 84.5 & 84.5 \\
    Average distance between minima, $\bar{y}$ (\si{cm}) \\
    $\bar{\lambda} = a\bar{y}/L$ (\si{nm}) \\
    Error $\Delta y$ in $\bar{y} = \sqrt{\sum_{i=1}^N (y_i - \bar{y})^2/(N-1)}$ (\si{cm}) \\
    Error $\Delta \lambda$ in $\bar{\lambda} = a \Delta y / L$ (\si{nm}) \\
    $\lambda = \bar{\lambda} \pm \Delta \lambda$ (\si{nm}) \\
  \end{tblr}
  \caption{Results of the first part of the experiment.}
  \label{tab:1}
\end{table}

\subsection*{Part B}

\begin{table}[ht]
  \centering
  \begin{tblr}{
    cells = {halign = c, valign = m},
    column{1} = {halign = l},
    row{odd} = {bg = lightgray!5},
    hlines = {},
    vlines = {}
  }
    Pattern & D & E & F \\
    \hline 
    Width of the slit, $a$ (\si{\mm}) & 0.5 & 0.025 & 0.0125 \\
    Distance from the slit to the screen, $L$ (\si{cm}) & 84.5 & 84.5 & 84.5 \\
    Average distance between minima, $\bar{y}$ (\si{cm}) \\
    $\bar{\lambda} = d\bar{y}/L$ (\si{nm}) \\
    Error $\Delta y$ in $\bar{y} = \sqrt{\sum_{i=1}^N (y_i - \bar{y})^2/(N-1)}$ (\si{cm}) \\
    Error $\Delta \lambda$ in $\bar{\lambda} = d \Delta y / L$ (\si{nm}) \\
    $\lambda = \bar{\lambda} \pm \Delta \lambda$ (\si{nm}) \\
  \end{tblr}
  \caption{Results of the second part of the experiment.}
  \label{tab:2}
\end{table}

\section{Discussion \& Conclusion}

\subsection*{Errors}

\subsection*{Approximations}

The experiment makes several approximations. Some of them are as follows:

\begin{itemize}
  \item \textbf{Fraunhofer diffraction pattern:} The experiment assumes that the diffraction pattern is in the far field, that is, the distance from the slit to the screen is much larger than the width of the slit. This assumption is necessary to simplify the calculations and to obtain a simpler diffraction pattern. 
  \item \textbf{Coherent light:} The experiment assumes that the light from the laser is coherent, that is, the light waves have the same frequency, wavelength, and constant relative phase. In reality, the light from the laser may not be perfectly coherent, which may affect the diffraction pattern.
\end{itemize}

\subsection*{Discrepancies}

\subsection*{Conclusion} 

\section{Extra Credit}

\td{Bro, if you once again write a one paragraph extra credit I will lose my mind.}

% \printbibliography

\end{document}