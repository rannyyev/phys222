\documentclass[10pt]{article}

% Default 
\usepackage{graphicx}
\usepackage[backend=biber,
  style=numeric, 
  sorting=none]{biblatex}

% Additional
\usepackage{amsmath}
\usepackage{textcomp, gensymb}
\usepackage{placeins}
\usepackage{tabularray} 
\usepackage{xcolor}
\usepackage{placeins}
\usepackage{csquotes} 
\usepackage{todonotes}
\usepackage{hyperref}
\usepackage{siunitx}

\newcommand{\td}[1]{\todo[linecolor=blue, backgroundcolor=blue!25,bordercolor=blue, size=\small, inline]{#1}}

\addbibresource{references.bib}

\title{Michelson Interferometer} 
\author{Rahmanyaz Annyyev, Hikmat Gulaliyev}
\date{28 April, 2024} 

\begin{document}

\maketitle

\begin{abstract}
  
\end{abstract}

\section{Introduction}

\subsection*{General}

One of the methods of producing interference patterns is amplitude division. In this method, a beam of light is divided into two beams, and the two beams are recombined to produce interference fringes. A Michelson interferometer is an instrument that employs this mechanism.

A Michelson interferometer is an instrument used in optical interferometry. Minimally, it consists of two mirrors $M_1$ and $M_2$ and a beam splitter $M$ (although a diffraction grating can be used). The setup is shown in Figure \ref{fig:1}. The beam splitter, in our case, is a plate beamsplitter with a partially reflective coating. A source of light, $S$, a laser in our case, is directed at the beam splitter, and the light is split into two beams of coherent light at point $C$. One beam is reflected toward fixed mirror $M_1$, and the other one is transmitted toward adjustable mirror $M_2$. The beams are reflected by the mirrors and recombined at the beam splitter at point $C'$. They then travel to a viewing screen where the interference pattern is observed (although a photoelectric detector or a camera can be used).

\begin{figure}[hbt!]
  \centering
  \includegraphics[scale=0.6]{figures/f1.pdf}
  \caption{The Michelson interferometer.}
  \label{fig:1}
\end{figure}

The beam of light that is produced by a He-Ne laser is too narrow for us to observe the interference fringes. To widen the beam, we use a lens, $L$, which is placed between the beam splitter and the screen. It spreads the beam, and the interference fringes become visible. However, only the central ray of the beam travels in a straight line. The rays that are further from the center travel at an angle and, therefore, have a different path length. This difference in path length causes the interference fringes to be curved. To make the fringes straight, we place the mirrors $M_1$ and $M_2$ at an angle of $90\degree$ to each other. This way, the rays that are reflected by the mirrors travel the same distance and the fringes are straight. 

The interference pattern will be a series of bright and dark concentric rings. The rings are formed because, as we stated earlier, as the rays are reflected by the mirrors, they travel different distances, and hence, have different relative phases. The relative phase difference includes the changes in phase due to the reflection of the light from the mirrors and the difference in the total path length the light travels. 

In this experiment, we are concerned with the phase changes. We can calculate the number of cycles of fringes that are produced as the mirror $M_2$ is moved. The amount of phase change is given by the formula
\begin{equation}
  \Delta \phi = 2 \pi \left(\dfrac{d}{\lambda/n}\right) = 2 \pi n \left(\dfrac{d}{\lambda}\right)
  \label{eq:1}
\end{equation}
where $d$ is the distance the mirror is moved, $\lambda$ is the wavelength of the light in a vacuum, and $n$ is the refractive index of the medium in which the light is traveling. Therefore, the interference pattern switches from bright to dark $N$ times as the mirror is moved a distance $d$ given by
\begin{equation}
  N \lambda = 2 n d
  \label{eq:2}
\end{equation}

\subsection*{Procedure}

The experiment is comprised of two parts: A and B. The setup consists of a PASCO{\textsuperscript\textregistered} interferometer, a lens with a holder, a viewing screen, a vacuum pump, a He-Ne laser, and a vacuum chamber.

\subsubsection*{Part A}

In the first part of the experiment, we will measure the wavelength of the light produced by the He-Ne laser. We will count the number of fringes, $N$, that pass a point on the screen as the mirror $M_2$ is moved a distance $d$. We will then use Equation \ref{eq:2} to calculate the wavelength of the light. Since $N$ can be quite large for a coherent light source like a laser, the precision of the measurement can be quite high. 

We first adjust the interferometer and the laser until a nearly circular, or elliptical, pattern is observed on the screen. Next, we adjust the micrometer knob so that the lever arm is nearly parallel with the edge of the interferometer base. In such a case, the relationship between the rotation of the knob and the movement of the mirror is almost linear. We turn the micrometer knob one full turn counterclockwise until the zero mark on the knob is aligned with the index mark. Next, we select a mark on the screen, slowly turn the micrometer counterclockwise, and observe the change in the interference pattern. Then, we count the number of fringes that pass the mark on the screen as we turn the knob. The final position of the knob and the distance $d$ the mirror has moved are recorded. At least 30 fringes should be counted, and the measurement should be repeated at least four times to obtain appreciable accuracy. All of the measurements should be recorded in a table.

\subsubsection*{Part B}

In the second part of the experiment, we will measure the refractive index of air. First, we align the laser and the interferometer just as we did in the first part of the experiment. Next, we fix the mirrors and place a vacuum chamber between the mirror $M_2$ and the beam splitter. The end plates of the vacuum chamber must be perpendicular to the axis of the interferometer for the best results. Then, we adjust the alignment screws of the mirror $M_1$ so that the center of the interference pattern is observed on the screen. We make a mark on the screen and evacuate the chamber.

As the air is pumped out of the chamber, the refractive index of the air will decrease, and the interference pattern will shift on the viewing screen. By counting the number of fringes that pass a point on the screen for different values of $P_{\text{final}}$, we can calculate the refractive index of the air using the formula
\begin{equation}
  n_{\text{atm}} = 1 + \dfrac{\left(N/\Delta P\right)}{\left(\lambda/2d\right)} P_{\text{atm}}
\end{equation}
where $n_{\text{atm}}$ is the refractive index of the air at atmospheric pressure, $N$ is the number of fringes that pass a point on the screen, $\Delta P$ is the change in pressure within the chamber, $P_{\text{atm}}$ is the atmospheric pressure, $\lambda$ is the wavelength of the light in vacuum, and $d$ is the thickness of the chamber. As an assumption, the temperature of the air is kept constant. 

\section{Data \& Results}

\subsection*{Part A}

The data obtained in the first part of the experiment is shown in Table \ref{tab:1}. The average wavelength of the light produced by the He-Ne laser is calculated using Equation \ref{eq:1}. It is assumed that the refractive index of air is $n = 1$. The average wavelength is calculated using the formula
\begin{equation}
  \bar{\lambda} = \dfrac{\sum_{i=1}^{n} \lambda_i}{n}
  \label{eq:3}
\end{equation}
and the standard deviation of the wavelength is calculated using the formula
\begin{equation}
  \Delta \lambda = \sqrt{\dfrac{\sum_{i=1}^{n} \left(\lambda_i - \bar{\lambda}\right)^2}{n-1}}
  \label{eq:4}
\end{equation}
where $n$ is the number of measurements. The wavelength of the light produced by the He-Ne laser is found to be $\lambda = \bar{\lambda} \pm \Delta \lambda$ nm.

\begin{table}[ht]
  \centering
  \vspace{4mm}
  \begin{tblr}{
    cells = {halign = c, valign = m},
    row{odd} = {bg = lightgray!5},
    row{1, 7} = {bg = lightgray!20},
    hlines = {},
    vlines = {}
  }
    $d_i$ (\si{\micro\metre}) & $N_i$ & $\lambda_i$ (nm) \\
    \hline 
    12 & 35 & 685.7 \\
    10.5 & 31 & 677.4 \\
    10 & 30 & 666.7 \\
    8.5 & 25 & 680.0 \\
    8 & 24 & 666.7 \\
    \hline
    $\bar{\lambda}$ (nm) & $\Delta \lambda$ (nm) & $\lambda$ (nm) \\
    675.3 & 8.4 & 675.3 $\pm$ 8.4
    
  \end{tblr}
  \caption{Results of the first part of the experiment.}
  \label{tab:1}
\end{table}

\subsection*{Part B}

The data obtained in the second part of the experiment is shown in Table \ref{tab:2}. The refractive index of air is calculated using Equation \ref{eq:2}. The thickness of the chamber was found to be $3.3$ cm The average refractive index of air is calculated using the formula
\begin{equation}
  \bar{n}_{\text{air}} = \dfrac{\sum_{i=1}^{n} n_{\text{air}_i}}{n}
  \label{eq:5}
\end{equation}
and the standard deviation of the refractive index is calculated using the formula
\begin{equation}
  \Delta n_{\text{air}} = \sqrt{\dfrac{\sum_{i=1}^{n} \left(n_{\text{air}_i} - \bar{n}_{\text{air}}\right)^2}{n-1}}
  \label{eq:6}
\end{equation}
where $n$ is the number of measurements. The refractive index of air is found to be $n_{\text{air}} = \bar{n}_{\text{air}} \pm \Delta n_{\text{air}}$.

\begin{table}[ht]
  \centering
  \vspace{4mm}
  \begin{tblr}{
    cells = {halign = c, valign = m},
    row{odd} = {bg = lightgray!5},
    row{1, 4} = {bg = lightgray!20},
    hlines = {},
    vlines = {}
  }
    $\Delta P_i$ (cmHg) & $N_i$ & $n_{\text{air}_i}$ \\
    \hline 
    20 & 10 & ? \\
    16 & 13 & ? \\
    \hline
    $\bar{n}_{\text{air}}$ & $\Delta n_{\text{air}}$ & $n_{\text{air}}$ \\
    
  \end{tblr}
  \caption{Results of the second part of the experiment.}
  \label{tab:2}
\end{table}

\section{Discussion \& Conclusion}

\subsection*{Errors}

\subsection*{Approximations}

\subsection*{Discrepancies}

\subsection*{Conclusion}

\section{Extra credit}

\td{Wikipedia page has a great section on applications of the Michelson interferometer. Use the info and the references: \newline \url{https://en.wikipedia.org/wiki/Michelson_interferometer}}

The Michelson interferometer was invented by an American physicist Albert Michelson in 1881. It is used in many areas of physics, including optics, astronomy, and metrology. It is used to measure the wavelength of light, the refractive index of a medium, and the thickness of thin films. It is also used in the detection of gravitational waves.

% \printbibliography

\end{document}